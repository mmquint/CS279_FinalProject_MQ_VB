In our study, we used low-cost, accessible technology to measure repetition count and form for a wide array of exercises. The inaccuracy and bias in our estimates point to important weaknesses in our model. 
First, it is clear that our methodology failed to accurately assess some exercise for most of our participants.
In particular, we got high bias and inaccuracy levels for arm circles.
An examination of the video footage revealed that this was due to a wide disparity in the range of arm movement among participants which caused a significant mismatch between the parameters we obtained from our training data and those that would be needed to accurately measure all participants. Indeed, it seems that our models would benefit from a dynamic, machine-learning-based component.
In contrast, the algorithm performed very well on exercises such as jumping jack where disparity in form has little impact on the usefulness of the parameters chosen due to the large extend of the exercise motion.

Second, a close examination of the videos against our accuracy data revealed that 
the accuracy of our predictions was to a large extent influenced by participants' form.
Indeed, much of the bias seemed to have been caused by poor or improper form causing false positive or false negative estimates.
In general, the better a participant' form, the more accurate the assessment generated by our code. 
This complicates the potential applicability of our tool in real life situations---it gives most accurate results to those who need improvement the least and comparatively misleading results to those who have a lot of room for improvement. It is precisely the latter group of users that we imagine would find the software we developed most useful. 

Third, form measurements could be greatly improved by including more data points.
For instance, proper of hand movement in jumping jacks could have been assessed.
While---as the experts we consulted assured us---arm movement has limited impact on the number of calories burned, it is still important in assessing form and we believe we could have attained greater measurement accuracy by taking it into account. 
Indeed, similarly to the exercisers they are designed to assess, our models have have room for improvement. We believe that more advanced signal processing methods would yield significant improvements both in eliminating bias and improving overall accuracy.