In our study, we used low-cost, accessible technology to measure repetition count and form for a wide array of exercises. The inaccuracy and bias in our estimates point to important weaknesses in our model. 
First, it is clear that our methodology failed to accurately assess some exercise for most of our participants.
In particular, we got high bias and inaccuracy levels for arm circles.
An examination of the video footage revealed that this was due to a wide disparity in the range of arm movement among participants which caused a significant mismatch between the parameters we obtained from our training data and those that would be needed to accurately measure all participants. Indeed, it seems that our models would benefit from a dynamic, machine-learning-based component.
In contrast, the algorithm performed very well on exercises such as squat jumps where disparity in form has little impact on the usefulness of the parameters chosen due to the large extend of the exercise motion.

Second, a close examination of the videos against our accuracy data revealed that 
the accuracy of our predictions was to a large extent influenced by participants' form.
Indeed, much of the bias seemed to have been caused by poor or improper form causing false positive or false negative estimates.
In general, the better a participant' form, the more accurate the assessment generated by our code. 
This complicates the potential applicability of our tool in real life situations---it gives most accurate results to those who need improvement the least and comparatively misleading results to those who have a lot of room for improvement. It is precisely the latter group of users that we imagine would find the software we developed most useful. In order for our system to be more applicable in real life scenarios, a learning component must also be included.  Additionally, it ought to be noted that although the granular data was preferred amongst the majority (75\%) of users, the participants expressed some confusion regarding how best to correctly interpret the results shown and how this relates to improvement in terms of form, stamina, and overall performance.

Third, form measurements could be greatly improved by including more data points.
For instance, the proper hand movement in jumping jacks could have been assessed.
While---as the experts we consulted assured us---arm movement has limited impact on the number of calories burned, it is still important in assessing form and we believe we could have attained greater measurement accuracy by taking into account arm movement, a data point we purposely excluded due to the an increase in the likelihood of the user "cheating" the system. 
Indeed, similarly to the exercisers they are designed to assess, our models have have room for improvement. We believe that more advanced signal processing methods would yield significant improvements both in eliminating bias and improving overall accuracy.

However, despite the above limitations of our data collection and processing, we believe the Microsoft Kinect was an effective piece of hardware and the Microsoft Kinect SDK was an effective software development environment.  The SDK supported the tracking of skeletal data, which is an effective method of collecting data as the system is blind to body fat composition and this therefore eliminates a potential for a system bias toward users with a lower body fat composition as well as decrease the chance of inaccurate measurement of users with a higher body fat composition.

Additionally, the sampling of form measurements and calculations proved to be rather successful.  In our system, we did not measure the form of every single exercise move, but rather a sample (3 exercise moves in total) of the moves to calculate the overall form percentage adjustment.  The form calculation sampling effectively distinguished participants with bad form who performed a higher volume of repetitions from the participants with good form who performed a relatively lower volume of repetitions.  From both the consultation and review discussions with the experts, it was made clear that an accurate system would prefer the participant with good form that performed a lower volume of repetitions, as this is more important in terms of stamina, caloric burn, and effectiveness of the workout routine.