The intensity of a MC type exercise movement can be measured by the amount of resistance (either body resistance or use of external weights), the volume of repetitions performed and the power or rate at which work is performed. It is our goal to use relatively inexpensive technology, such as the Microsoft Kinect, to measure exerciser’s progress with a reasonable accuracy.
We aim to meet with our fitness experts to create a workout in the style of P90x and Insanity that features exercises that fit the following criteria:
\begin{itemize}
	\item Requires little to no prior skill or training 
	\item Requires strength to accurately achieve the best stance for that particular move
	\item Requires stamina to complete the exercise move at a fast rate.
\end{itemize}
We will use the performance of our experts with the generated MC workout routine and use these accumulation of these and a first set of participants' performance to establish the expert fitness benchmark.  We then plan to both record the footage of each participant completing the routine and use the depth sensor to measure factors such as rate of exercise move, body form (predominantly angles between limbs), speed, height (where applicable), the duration and frequency of resting time, and the volume of repetitions performed.  We will then calculate a score or percentage of the expert benchmark using the collected data as well as generating more granular data specific to each workout move completed.  After the participant has finished the workout, we will show them the two different reports 1.) The expert percentage score and 2.) The granular exercise move specific report and ask them to determine which report would be most helpful for them in terms of tracking their performance from workout to workout.