Currently, multiple attempts have been made to increase exercise adherence and motivation through feedback provided by varying technologies and interfaces.  Metabolic Conditioning routines currently lack equally progressive and innovative methods of receiving post-workout feedback.  The purpose of our study was to develop an exercise-tracking tool capable of tracking the volume and rate of exercise moves completed during a specified time period, as well as the quality or form of the move, for a wide variety of exercises in order to determine which type of feedback users would find most helpful in tracking their progress over time.  We recruited 20 participants to use our system in order for us to both track the accuracy of our fitness performance models and to receive feedback regarding the differing performance reports shown to the participant post-workout.  Our findings show that the majority of users (75\%) preferred a detailed report of their performance as opposed to a holistic one.  In addition to recording the preference feedback of our users, we analyzed the video data as well as enlisted the help of fitness experts to gauge the success of our system.  This paper outlines an attempt at using inexpensive and accessible hardware and software to provide post-workout feedback to users participating in metabolic conditioning workout routines.