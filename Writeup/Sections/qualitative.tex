In order to measure the qualitative and quantitative accuracy of our performance models, we enlisted the help of 3 experts in the field of fitness that are currently employed by the Harvard athletic department (see acknowledgments for specific titles).   Although participant was assigned an expert score, the score itself was not as important to the ranking model as was the ability to successfully rank users relatively based on performance.  For this reason, the accuracy of our ranking model was tested in terms of each participant's overall ranking as opposed to the expert score.\\
 For each of the four performances regarding form (Jumping jacks, high marches, squat jumps, and overall performance adjusted by form), the 20 participants were sorted in order of descending performance score.  Groupings were then formed based on rank and the percentage difference in ranking.  The three ranking pairs chosen were the following: \\
\begin{itemize}
	\item{1st and 20th, 95\% difference in ranking} \\
	\item{5th and 16th, 55\% difference in ranking} \\
	\item{9th and 12th, 15\% difference in ranking} 
\end{itemize}
For the three specific exercise moves, the experts were shown 30 seconds of footage for each of the two participants performing the move in the ranking pair.  The experts were then asked to indicate which participant had the better form for that specific move.  In order to judge the best performance adjusted for form, the experts were shown 30 seconds of each of the participants performing the moves not specified (Arm circles, Knee to Elbows, Side to Sides, and Tuck Jumps) and then asked to indicate which participant had the better form overall based on what they were shown.  The following sections discuss the results from the expert session in order of Jumping Jacks, High Marches, Squat Jumps, and Overall Form respectively.\\

\subsection{Jumping Jacks Results}
While the experts were determining the top performer in each of the ranking pairs, they were told to give as much or as little feedback as they saw fit.  The first expert indicated that they did not believe any of the footage they were shown of participants completing jumping jacks demonstrated proper form, and therefore they selected between the ranking pairs in terms of how effectively the participant used both their legs and their arms to perform the move.  The model created to analyze form in jumping jacks did not track the form of the participants arms, but rather how effectively they used their legs.  This would perhaps explain the discrepancy in the ratings from expert 1 in terms of how their assessments matched those calculated by the system, which is ultimately the main goal of this portion of our analysis.  Expert 2 also did not find the form of the participants to be proper or highly effective.  Like expert 1, this expert distinguished between the pairs by judging the range of motion demonstrated as well as the form and extension of the arms. Expert 3 did not have any specific comments regarding the method utilized to distinguish between the pairs.  \\
See table 1 to see the expert ranking pair results.  The first column denotes the two rankings in each ranking pair (represented by a line) and the second, third, and fourth columns denote the chosen participant in the ranking pair by the first, second, and third expert respectively.  The fifth row denotes the number of ranking pair matches meaning that the expert ranked the two participants in that particular pair the same as the ranking outputted by the system based upon the relative performance of the two participants.  The ranking pair matches (case in which the system generated relative ranking matches that of the expert) are in bold font.\\

\begin{table}[h!]
\caption{Jumping Jacks Results}
\centering
\begin{tabular}{c c c c}
\hline \hline
Ranking Pair & 1 and 20 & 5 and 16 & 9 and 12 \\ [0.5ex]
\hline
Expert \#1 &		20		  			&16		  				&\boxed{\textbf{9}} \\
Expert \#2 &		\boxed{\textbf{1}}	&\boxed{\textbf{5}}		&\boxed{\textbf{9}} \\
Expert \#3 &		20		  			&\boxed{\textbf{5}}		&\boxed{\textbf{9}} \\
\hline 
\textbf{\# Correct Rankings} &		\textbf{1}&		\textbf{2}&		\textbf{3} \\
\end{tabular}
\label{table:jumpingjacksresult}
\end{table}

\subsection{High Marches Results}
Only expert 3 commented on their selection process during the High Marches form comparison.  This expert indicated that they chose those that had straighter and higher legs as this would directly reflect the flexibility of the participants and therefore the effectiveness of the move.  This corresponds with the model used to analyze form from within the system.\\
See table 2 to see the expert ranking pair results.  The first column denotes the two rankings in each ranking pair (represented by a line) and the second, third, and fourth columns denote the chosen participant in the ranking pair by the first, second, and third expert respectively.  The fifth row denotes the number of ranking pair matches meaning that the expert ranked the two participants in that particular pair the same as the ranking outputted by the system based upon the relative performance of the two participants. The ranking pair matches (case in which the system generated relative ranking matches that of the expert) are in bold font.\\

\begin{table}[h!]
\caption{High Marches Results}
\centering
\begin{tabular}{c c c c}
\hline \hline
Ranking Pair & 1 and 20 & 5 and 16 & 9 and 12 \\ [0.5ex]
\hline
Expert \#1 &		\boxed{\textbf{1}}	&\boxed{\textbf{5}}		&12 \\
Expert \#2 &		\boxed{\textbf{1}}	&\boxed{\textbf{5}}		&12 \\
Expert \#3 &		\boxed{\textbf{1}}	&\boxed{\textbf{5}}		&12 \\
\hline 
\textbf{\# Correct Rankings} &		\textbf{3}&		\textbf{3}&		\textbf{0} \\
\end{tabular}
\label{table:jumpingjacksresult}
\end{table}

\subsection{Squat Jumps Results}
For this move in particular, the participants were not given instructions regarding what to do with the arms during the execution of the move, therefore, experts were told to ignore this element of the move in terms of form as it would not be fair for extraneous arm positions to count against a participant if they were not given explicit instructions.  In terms of the comments collected, expert 1 indicated that they compared the participants in terms of how controlled the move appeared.  Expert 2 judged based the depth of the squat portion of the move and the perceived effort level in terms of accuracy in performing the move.  Expert 3 also judged based upon the perceived effort level.\\
See table 1 to see the expert ranking pair results.  The first column denotes the two rankings in each ranking pair (represented by a line) and the second, third, and fourth columns denote the chosen participant in the ranking pair by the first, second, and third expert respectively.  The fifth row denotes the number of ranking pair matches meaning that the expert ranked the two participants in that particular pair the same as the ranking outputted by the system based upon the relative performance of the two participants. The ranking pair matches (case in which the system generated relative ranking matches that of the expert) are in bold font.\\

\begin{table}[h!]
\caption{Squat Jumps Results}
\centering
\begin{tabular}{c c c c}
\hline \hline
Ranking Pair & 1 and 20 & 5 and 16 & 9 and 12 \\ [0.5ex]
\hline
Expert \#1 &		\boxed{\textbf{1}}	&\boxed{\textbf{5}}		&\boxed{\textbf{9}} \\
Expert \#2 &		\boxed{\textbf{1}}	&\boxed{\textbf{5}}		&12 \\
Expert \#3 &		20					&\boxed{\textbf{5}}		&\boxed{\textbf{9}} \\
\hline 
\textbf{\# Correct Rankings} &		\textbf{2}&		\textbf{3}&		\textbf{2} \\
\end{tabular}
\label{table:jumpingjacksresult}
\end{table}

\subsection{Overall Form Results}
In order to distinguish between the ranking pairs in this category, the experts were shown footage of each move (not shown above) the participant completed and asked to form a ranking based upon the overall form.\\
See table 1 to see the expert ranking pair results.  The first column denotes the two rankings in each ranking pair (represented by a line) and the second, third, and fourth columns denote the chosen participant in the ranking pair by the first, second, and third expert respectively.  The fifth row denotes the number of ranking pair matches meaning that the expert ranked the two participants in that particular pair the same as the ranking outputted by the system based upon the relative performance of the two participants. The ranking pair matches (case in which the system generated relative ranking matches that of the expert) are in bold font.\\

%\begin{table}[h!]
%\caption{Overall Form Results}
%\centering
%\begin{tabular}{c c c c}
%\hline \hline
%Ranking Pair & Expert \#1 & Expert \#2 & Expert \#3 \\ [0.5ex]
%\hline
%1 and 20 &		\boxed{1}&		\boxed{1}&		\boxed{1}\\
%5 and 16 &		\boxed{5}&		\boxed{5}&		\boxed{5}\\
%9 and 12 &		\boxed{9}&		\boxed{9}&		\boxed{9} \\
%\hline
%\textbf{\# Ranking Pair Matches} &		\textbf{3}&		\textbf{3}&		\textbf{3}
%\end{tabular}
%\label{table:overallformresult}
%\end{table}

\begin{table}[h!]
\caption{Overall Form Results}
\centering
\begin{tabular}{c c c c}
\hline \hline
Ranking Pair & 1 and 20 & 5 and 16 & 9 and 12 \\ [0.5ex]
\hline
Expert \#1 &		\boxed{\textbf{1}}	&\boxed{\textbf{5}}		&\boxed{\textbf{9}} \\
Expert \#2 &		\boxed{\textbf{1}}	&\boxed{\textbf{5}}		&\boxed{\textbf{9}} \\
Expert \#3 &		\boxed{\textbf{1}}	&\boxed{\textbf{5}}		&\boxed{\textbf{9}} \\
\hline 
\textbf{\# Correct Rankings} &		\textbf{3}&		\textbf{3}&		\textbf{3} \\
\end{tabular}
\label{table:jumpingjacksresult}
\end{table}
