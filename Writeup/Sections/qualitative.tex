In order to measure the qualitative accuracy of our performance models, we enlisted the help of 3 experts in the field of fitness that are currently employeed by the Harvard athletic department.  One of the most important goals of our project was to succeed in terms of relative performance ranking, i.e. the expert score percentage was not as important as the ability to successfully compare the form and performance of participants in relation to each other.  For each of the four performances regarding form (Jumping jacks, high marches, squat jumps, and overall performance adjusted by form), the 20 participants were sorted in order of descending performance score.  Groupings were then formed based on rank and the percentage difference  in ranking.  The three ranking pairs choosen were the following: \\
\begin{itemize}
	\item{1st and 20th, 95\% difference in ranking} \\
	\item{5th and 16th, 55\% difference in ranking} \\
	\item{9th and 12th, 15\% difference in ranking} 
\end{itemize}
For the three specific exercise moves, the experts were shown 30 seconds of each of the two participants performing the move in the ranking pair and asked to indicate which participant had the better form for that specific move.  In order to judge the best performance adjusted for form, the experts were shown 30 seconds of each of the participants performing the moves not specified (Arm circles, Knee to Elbows, Side to Sides, and Tuck Jumps) and then asked to indicate which participant had the better form overall based on what they were shown.  The following four tables demonstrate the results from the expert session in order of Jumping Jacks, High Marches, Squat Jumps, and Overall Form respectively.\\

\begin{table}[h!]
\caption{Jumping Jacks Results}
\centering
\begin{tabular}{c c c c}
\hline \hline
Ranking Pair & Expert \#1 & Expert \#2 & Expert \#3 \\ [0.5ex]
\hline
1 and 20 &		20&		1&		20 \\
5 and 16 &		16&		5&		5 \\
9 and 12 &		9&		9&		9 \\
\hline
\# Correct (out of 3)&		0&		3&		2 \\
\end{tabular}
\label{table:jumpingjacksresult}
\end{table}

\begin{table}[h!]
\caption{High Marches Results}
\centering
\begin{tabular}{c c c c}
\hline \hline
Ranking Pair & Expert \#1 & Expert \#2 & Expert \#3 \\ [0.5ex]
\hline
1 and 20 &		1&		1&		1 \\
5 and 16 &		5&		5&		5 \\
9 and 12 &		12&		12&		12 \\
\hline
\# Correct (out of 3)&		2&		2&		2 \\
\end{tabular}
\label{table:highmarchesresult}
\end{table}

\begin{table}[h!]
\caption{Squat Jumps Results}
\centering
\begin{tabular}{c c c c}
\hline \hline
Ranking Pair & Expert \#1 & Expert \#2 & Expert \#3 \\ [0.5ex]
\hline
1 and 20 &		1&		1&		20 \\
5 and 16 &		5&		5&		5 \\
9 and 12 &		9&		12&		9 \\
\hline
\# Correct (out of 3)&		3&		2&		2 \\
\end{tabular}
\label{table:squatjumpsresult}
\end{table}

\begin{table}[h!]
\caption{Overall Form Results}
\centering
\begin{tabular}{c c c c}
\hline \hline
Ranking Pair & Expert \#1 & Expert \#2 & Expert \#3 \\ [0.5ex]
\hline
1 and 20 &		1&		1&		1 \\
5 and 16 &		5&		5&		5 \\
9 and 12 &		9&		9&		9 \\
\hline
\# Correct (out of 3)&		3&		3&		3 \\
\end{tabular}
\label{table:overallformresult}
\end{table}