The notion of tracking one's progress and quantifying improvement is an old one. As early as the 18th century, Benjamin Franklin kept a daily journal to monitor his performance in thirteen personal virtues. 
% CITATION
%%%
Modern research provides formal basis for what Benjamin Franklin seemed to know intuitively: the idea that measurement improves adherence and ultimately promotes long-term behavioral change. For example, Dr. Melody Noland measured the effects of self-monitoring and reinforcement on adherence to unsupervised exercise
\cite{Noland1989Effects}. 
Splitting the study subjects into three groups (self-monitoring, reinforcement supplied by another person, and control), she found that the self-monitoring and reinforcement groups reported a significantly higher frequency of exercise and better results than the control group 
\cite{Noland1989Effects}. % CITATION
%%%
Despite the encouraging results, Dr. Noland's methodology may today seem outdated: the self-monitoring subjects were required to keep a written log of their exercise
\cite{Noland1989Effects}, % CITATION
leaving much to be desired in terms of usability and automation of what may otherwise be a very tedious process. Indeed, the decade following the 1989 publication of Dr. Nolan’s research witnessed the onset of persuasive technologies, which likewise leveraged progress tracking as a means of sustaining desired behavior. 
%%%
One of the pioneers of the field, Dr. Sunny Consolvo and her co-authors explored the ways in which technology could be used to encourage behavioral change
\cite{Consolvo2009Theory, Consolvo2008Flowers}. % CITATION
In ``Theory-Driven Design Strategies for Technologies that Support Behavior Change in Everyday Life", Consolvo et. al. did an excellent job drawing on prior research in behavioral psychology and situating their work in the context of relevant psychology theories. For instance, the Goal-Setting Theory stresses the importance of an ongoing feedback as progress is made as opposed to merely providing a reward when a goal is achieved 
\cite{Consolvo2009Theory}. % CITATION
Choosing data abstraction as one of the main design tenets of UbiFit, the software they created for the purposes of their study, Consolvo et. al. implemented an animated garden that tracks and encourages progress toward the desired goal 
\cite{Consolvo2009Theory, Consolvo2008Flowers}. % CITATION
%%%
Our research takes a different approach by instead enabling the measurement of granular, incremental changes that the UbiFit garden interface does not capture. As such, our research enters a contented field of performance measurement tracking devices. Indeed, those interested in physical activity tracking devices have a myriad of options, such as Jawbone Up, Fitbit, or the Nike Fuel Band, all of which offer different ways of performance tracking. Relying predominantly on accelerometers, theses devices fail to account for improvements in form and are prone to count mindless handshaking toward the user's exercise score with much opportunity for the user to game the system. Unlike these commercial devices, our goal is to use Microsoft Kinect to develop exercise-tracking software that could accurately measure the incremental qualitative improvements in a person's form and overall execution of an exercise routine, rather than a raw quantity of physical activity in insolation.

