The notion of tracking one's progress and quantifying improvement is an old one. As early as the 18th century, Benjamin Franklin kept a daily journal to monitor his performance in thirteen personal virtues \cite{BenjaminFranklin}. Modern research provides formal basis for what Benjamin Franklin seemed to know intuitively: the idea that measurement improves adherence and may ultimately promote long-term behavioral change. For example, Melody Noland showed that self-monitoring improves frequency and adherence to unsupervised exercise \cite{Noland1989Effects}. Despite the encouraging results, Noland's methodology may today seem outdated: the self-monitoring subjects were required to keep a written log of their workouts \cite{Noland1989Effects}, leaving much to be desired in terms of usability and automation of what may otherwise be a very tedious process. Indeed, in a comprehensive study with over 160 participants, James J. Annesi demonstrated that automated, computer-based feedback system leads to better exercise adherence and attendance as well as lower dropout rates compared to traditional paper-and-pencil exercise-tracking system \cite{Annesi1998}.

As if in response to Annesi's 1998 study, the past decade has witnessed the onset of technologies that automate exercise tracking, particularly by using human motion capture. Tools were developed to measure exercise form as well as its frequency. Focusing on exercise form, Havens et. al. developed a video-based motion tracking system to provide feedback about posture and stability of elders during exercise \cite{Havens2009}, and De Silva et. al. used Body Sensor Network (BSN) to provide real-time corrective feedback during the performance of an exercise routine \cite{DeSilva2008}. Focusing on exercise frequency, Chang et. al. used accelerometer embedded in workout gloves to detect different weight training exercises and compute the rate at which they were performed \cite{Chang2007}.

Despite the great advances in exercise measurement, limited attention has been given to the type of feedback users would find most helpful.  Indeed, the type of feedback researchers provided to their participants varied widely: from granular ones such as the one supplied by \cite{Annesi1998} or Havens et. al.  \cite{Havens2009} to holistic ones such as the one Sunny Consolvo and her co-authors created when they sought to display a participant's progress in the form of an animated garden \cite{Consolvo2009Theory}. Similar dichotomy exists in the industry. While some products---such as Fitbit \cite{Fitbit} or Jawbone UP \cite{Jawbone}� seek to provide detailed overview of a user's activity, from distance travelled and stairs climbed to the number of hours slept, others---such as the Nike FuelBand---seek to abstract away the wearer's daily activity into a single metric, the so-called "Fuel Points`` \cite{Nike}. Accordingly, the purpose of our research was to employ affordable technology to accurately track both the form and rate of a wide range of exercises, and use the output to investigate which type of feedback---holistic or granular---users would find most helpful as a means of tracking progress and improvement over time.