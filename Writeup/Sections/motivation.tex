Intensive workout programs such as P90x, Insanity, and Rushfit promise users to transform their bodies into the “best shape of their life” by following a 60 to 90 days workout regimen.  These types of routines feature methodologies such as “Muscle Confusion,” “Max Interval Training,” or “High Intensity Interval Training,” which all fall into the Metabolic Conditioning exercise category---exercises that increase the storage and delivery of energy for any activity through the improvement of strength and endurance.  While a multitude of users report great results, many do not endure long enough to reap the fitness benefits of these programs. Prior research (see related work section) has shown that motivation improves exercise adherence. Accordingly, it is our belief that one of the main reasons people fail to persevere these exercise regimen is the lack of observable progress from one workout to the next, a crucial motivating factor. Unlike strength training exercises--such as bench press or deadlift--the incremental improvements of Metabolic Conditioning workouts are often not apparent until weeks into the program, causing many users to lose motivation and quit the program early in the 60-90 days regimen. The goal of our study is to leverage motion tracking devices such as the Microsoft Kinect to create a software tool for calculating incremental progress for Metabolic Conditioning exercises and ultimately motivate users to adhere to their Metabolic Conditioning routine of choice.