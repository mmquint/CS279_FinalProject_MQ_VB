Despite the limitations mentioned in the discussion section, we see our results as very encouraging. Indeed, as our research indicates, users of many different levels of athletic ability found detailed, computer generated exercise feedback more helpful than an aggregate metric; we demonstrated the feasibility of creating such as tool with affordable hardware.  However, we would recommend future attempts include a teaching component that better outlines the results and provides detailed instructions regarding how to best improve in terms of performance, form and effectiveness of the individual exercise moves.  Additionally, our research suggested that user-blind form analysis sampling could be an effective and less expensive (in terms of developer resources and time)  method for calculating overall form performance.\\
In the future, software tools and inexpensive hardware, such as the Microsoft Kinect, could provide a powerful incentive to engage in  metabolic conditioning and potentially increase exercise routine adherence.  More broadly, this research seeks to inspire future work and improved methods for tracking and analyzing exercise performance in real time of a broader spectrum of exercise types, as there is currently a bias towards cardiovascular activity, such as running and walking, and exercise practices that require much more expert advice through teaching in addition to skill, such as yoga.  Another important element of our research is to inspire further analysis of the potential types of feedback that could be provided to users regarding their fitness performance.



