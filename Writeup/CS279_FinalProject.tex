\documentclass{sigchi}

% Use this command to override the default ACM copyright statement (e.g. for preprints). 
% Consult the conference website for the camera-ready copyright statement.


%% EXAMPLE BEGIN -- HOW TO OVERRIDE THE DEFAULT COPYRIGHT STRIP -- (July 22, 2013 - Paul Baumann)
% \toappear{Permission to make digital or hard copies of all or part of this work for personal or classroom use is 	granted without fee provided that copies are not made or distributed for profit or commercial advantage and that copies bear this notice and the full citation on the first page. Copyrights for components of this work owned by others than ACM must be honored. Abstracting with credit is permitted. To copy otherwise, or republish, to post on servers or to redistribute to lists, requires prior specific permission and/or a fee. Request permissions from permissions@acm.org. \\
% {\emph{CHI'14}}, April 26--May 1, 2014, Toronto, Canada. \\
% Copyright \copyright~2014 ACM ISBN/14/04...\$15.00. \\
% DOI string from ACM form confirmation}
%% EXAMPLE END -- HOW TO OVERRIDE THE DEFAULT COPYRIGHT STRIP -- (July 22, 2013 - Paul Baumann)


% Arabic page numbers for submission. 
% Remove this line to eliminate page numbers for the camera ready copy
\pagenumbering{arabic}


% Load basic packages
\usepackage{balance}  % to better equalize the last page
\usepackage{graphics} % for EPS, load graphicx instead
\usepackage{times}    % comment if you want LaTeX's default font
\usepackage{url}      % llt: nicely formatted URLs

% llt: Define a global style for URLs, rather that the default one
\makeatletter
\def\url@leostyle{%
  \@ifundefined{selectfont}{\def\UrlFont{\sf}}{\def\UrlFont{\small\bf\ttfamily}}}
\makeatother
\urlstyle{leo}


% To make various LaTeX processors do the right thing with page size.
\def\pprw{8.5in}
\def\pprh{11in}
\special{papersize=\pprw,\pprh}
\setlength{\paperwidth}{\pprw}
\setlength{\paperheight}{\pprh}
\setlength{\pdfpagewidth}{\pprw}
\setlength{\pdfpageheight}{\pprh}

% Make sure hyperref comes last of your loaded packages, 
% to give it a fighting chance of not being over-written, 
% since its job is to redefine many LaTeX commands.
\usepackage[pdftex]{hyperref}
\hypersetup{
pdftitle={SIGCHI Conference Proceedings Format},
pdfauthor={LaTeX},
pdfkeywords={SIGCHI, proceedings, archival format},
bookmarksnumbered,
pdfstartview={FitH},
colorlinks,
citecolor=black,
filecolor=black,
linkcolor=black,
urlcolor=black,
breaklinks=true,
}

% create a shortcut to typeset table headings
\newcommand\tabhead[1]{\small\textbf{#1}}


% End of preamble. Here it comes the document.
\begin{document}

\title{Striving for Perfection: Measurement of Incremental Fitness Improvement}

\numberofauthors{2}
\author{
  \alignauthor Vladimir Bok\\
    \affaddr{Harvard University}\\
    \affaddr{117 Eliot Mail Center, Cambridge, MA 02138}\\
    \email{Vladimirbok@college.harvard.edu}
  \alignauthor Meg Quintero\\
    \affaddr{Harvard University}\\
    \affaddr{514 Kirkland Mail Center, Cambridge, MA 02138}\\
    \email{mmquint@college.harvard.edu}
}

\maketitle

\begin{abstract}
TBA
\end{abstract}

\keywords{
	Kinect; Metabolic Conditioning;
}

\section{Introduction}
\subsection{Motivation}
Intensive workout programs such as P90x, Insanity, and Rushfit promise users to transform their bodies into the “best shape of their life” by following a 60 to 90 days workout regimen.  These types of routines feature methodologies such as “Muscle Confusion,” “Max Interval Training,” or “High Intensity Interval Training,” which all fall into the Metabolic Conditioning exercise category---exercises that increase the storage and delivery of energy for any activity through the improvement of strength and endurance.  While a multitude of users report great results, many do not endure long enough to reap the fitness benefits of these programs. Prior research (see related work section) has shown that motivation improves exercise adherence. Accordingly, it is our belief that one of the main reasons people fail to persevere these exercise regimen is the lack of observable progress from one workout to the next, a crucial motivating factor. Unlike strength training exercises--such as bench press or deadlift--the incremental improvements of Metabolic Conditioning workouts are often not apparent until weeks into the program, causing many users to lose motivation and quit the program early in the 60-90 days regimen. The goal of our study is to leverage motion tracking devices such as the Microsoft Kinect to create a software tool for calculating incremental progress for Metabolic Conditioning exercises and ultimately motivate users to adhere to their Metabolic Conditioning routine of choice.

\subsection{Approach}
The intensity of a MC type exercise movement can be measured by the amount of resistance (either body resistance or use of external weights), the volume of repetitions performed and the power or rate at which work is performed. It is our goal to use relatively inexpensive technology, such as the Microsoft Kinect, to measure exerciser’s progress with a reasonable accuracy.
We aim to meet with our fitness experts to create a workout in the style of P90x and Insanity that features exercises that fit the following criteria:
\begin{itemize}
	\item Requires little to no prior skill or training
	\item Requires strength to accurately achieve the best stance for that particular move
	\item Requires stamina to complete the exercise move at a fast rate.
\end{itemize}
We will use the performance of our experts with the generated MC workout routine and use these accumulation of these to establish the expert fitness benchmark.  We then plan to both record the footage of each participant completing the routine and use the depth sensor to measure factors such as rate of exercise move, body form (predominantly angles between limbs), speed, height (where applicable), the duration and frequency of resting time, and the volume of repetitions performed.  We will then calculate a score or percentage of the expert benchmark using the collected data.  We will track the progress of participant over a span of 14 days and after the completion of each session we will gather the participant’s mental recollection of their progress before and after the progress data is revealed to them using both in person interview style questions and a questionnaire to collect a wide array of both qualitative and quantitative data and to avoid bias.

\subsection{Contribution}
If successful, our work will show that currently hard to benchmark exercises can be evaluated in terms of performance with relatively inexpensive hardware.  We predict that our measurement tool could be meaningful to potential users to track minute performance improvements and, ultimately, inspire them to continue working towards their fitness goals.  Much research has already been done regarding measuring exercise performance using technology that has inspired and motivated this research project.  We believe our work addresses a hole in this field that has not been explored that we believe will spark future work in the field of exercise motivation and progress analysis through the use of inexpensive technology.
\pagebreak
\section{Previous Work}
%%%
The notion of tracking one's progress and quantifying improvement is an old one. As early as the 18th century, Benjamin Franklin kept a daily journal to monitor his performance in thirteen personal virtues. 
%%%
Modern research provides formal basis for what Benjamin Franklin seemed to know intuitively: the idea that measurement improves adherence and ultimately promotes long-term behavioral change. For example, Dr. Melody Noland measured the effects of self-monitoring and reinforcement on adherence to unsupervised exercise
\cite{Noland1989Effects}. 
Splitting the study subjects into three groups (self-monitoring, reinforcement supplied by another person, and control), she found that the self-monitoring and reinforcement groups reported a significantly higher frequency of exercise and better results than the control group 
\cite{Noland1989Effects}. % CITATION
%%%
Despite the encouraging results, Dr. Noland's methodology may today seem outdated: the self-monitoring subjects were required to keep a written log of their exercise
\cite{Noland1989Effects}, % CITATION
leaving much to be desired in terms of usability and automation of what may otherwise be a very tedious process. Indeed, the decade following the 1989 publication of Dr. Nolan’s research witnessed the onset of persuasive technologies, which likewise leveraged progress tracking as a means of sustaining desired behavior. 
%%%
One of the pioneers of the field, Dr. Sunny Consolvo and her co-authors explored the ways in which technology could be used to encourage behavioral change
\cite{Consolvo2009Theory, Consolvo2008Flowers}. % CITATION
In ``Theory-Driven Design Strategies for Technologies that Support Behavior Change in Everyday Life", Consolvo et. al. did an excellent job drawing on prior research in behavioral psychology and situating their work in the context of relevant psychology theories. For instance, the Goal-Setting Theory stresses the importance of an ongoing feedback as progress is made as opposed to merely providing a reward when a goal is achieved 
\cite{Consolvo2009Theory}. % CITATION
Choosing data abstraction as one of the main design tenets of UbiFit, the software they created for the purposes of their study, Consolvo et. al. implemented an animated garden that tracks and encourages progress toward the desired goal 
\cite{Consolvo2009Theory, Consolvo2008Flowers}. % CITATION
%%%
Our research takes a different approach by instead enabling the measurement of granular, incremental changes that the UbiFit garden interface does not capture. As such, our research enters a contented field of performance measurement tracking devices. Indeed, those interested in physical activity tracking devices have a myriad of options, such as Jawbone Up, Fitbit, or the Nike Fuel Band, all of which offer different ways of performance tracking. Relying predominantly on accelerometers, theses devices fail to account for improvements in form and are prone to count mindless handshaking toward the user's exercise score with much opportunity for the user to game the system. Unlike these commercial devices, our goal is to use Microsoft Kinect to develop exercise-tracking software that could accurately measure the incremental qualitative improvements in a person's form and overall execution of an exercise routine, rather than a raw quantity of physical activity in insolation.

\section{Conclusion}

TBA

\section{Acknowledgments}
\section{References format}
References must be the same font size as other body text.
% REFERENCES FORMAT
% References must be the same font size as other body text.

\bibliographystyle{acm-sigchi}
\bibliography{sample}
\end{document}
